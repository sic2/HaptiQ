\documentclass[a4paper]{article}

\usepackage[english]{babel}
\usepackage[utf8]{inputenc}
\usepackage{amsmath}
\usepackage{graphicx}
\usepackage[colorinlistoftodos]{todonotes}

\usepackage{hyperref}
\hypersetup{
    colorlinks,
    citecolor=black,
    filecolor=black,
    linkcolor=black,
    urlcolor=black
}

% Allow C# Code to be embedded
\usepackage{listings}
\usepackage{xcolor}
\renewcommand{\lstlistingname}{Code}

\lstdefinestyle{sharpc}
{language=[Sharp]C, 
frame=single, 
rulecolor=\color{blue!80!black}, 
keywordstyle=\color{blue}, 
numbers=left,
breaklines=true,
basicstyle=\footnotesize\singlespacing}

\lstdefinestyle{sharpc1}
{language=[Sharp]C, 
frame=single, 
rulecolor=\color{blue!80!black}, 
keywordstyle=\color{blue}, 
numbers=none,
breaklines=true,
basicstyle=\footnotesize\singlespacing}

\title{HaptiQ Manual}

\author{Simone Ivan Conte \\\href{mailto:sic2@st-andrews.ac.uk}{sic2@st-andrews.ac.uk}}

\begin{document}
\maketitle

\newpage

\tableofcontents
\newpage

\section{Introduction}

This manual explains how to create WPF applications using the HaptiQ API and how to print the HaptiQ devices. \\
For further information or help, please contact the author of this manual.

\section{API}

This section will explain how to create a basic WPF application using the HaptiQ API. 

\subsection{Create a simple WPF project}

\begin{itemize}
	\item how to create WPF project
    \item add references
    \item change xml file
    \item import (using) HaptiQ API
\end{itemize}


\todo[inline, color=green!40]{describe step by step}
The API contains a multitude of functionality. For more information it is suggested to use Visual Studio, which allows easy access to the documentation for the API. 

\subsection{Extend the HaptiQ API}

\subsection{Extend the Input API}

\todo[inline, color=green!40]{describe step by step}

\section{Hardware}

The models for the 4-HaptiQ and the 8-HaptiQ can be found in the folder \textit{HaptiQ 3D Models}. Use MakerWare to import the STL files and print all the interested parts. Note that the printing can take up to several hours, based on the 3D printer used. 

\subsection{Hardware needed}

\begin{itemize}
	\item 3D Printer with 1.8mm ABS Plastic or equivalent
    \item 4 Hitech HS-65MG micro servomotors (8 for the 8-HaptiQ)
    \item 1 Phidget AdvancedServoBoard
    \item 1 Phidget InterfaceKit 8/8/8 or equivalent (must support analog input)
    \item 4 Force-Sensing resistors (8 for the 8-HaptiQ)
    \item Screws of various dimensions 
    \item Super glue
    \item Female and male crimps (with slots)
\end{itemize}

\section{Software Listing}

The software used in this project is listed below:

\begin{center}
    \begin{tabular}{ | l | p{3cm} | p{8cm} |}
    \hline
    Software & Description & Available at \\ \hline
    Phidget libraries & Control servos and pressure sensors & http://www.phidgets.com/ \\ \hline
    GRATF & Locate and recognize Glyphs & http://www.aforgenet.com/projects/gratf/ \\ \hline
    Glyphs Studio & Print Glyphs & http://www.aforgenet.com/projects/gratf/ \\ \hline
    Surface SDK 2.0 & Interface with tabletop & http://www.microsoft.com/en-gb/download/details.aspx?id=26716 \\ \hline
    NCalc & Evaluate mathematical functions in FunctionsApp & http://ncalc.codeplex.com/ \\ \hline
    VS2012 & C\# IDE & \- \\ \hline
    MakerWare & 3D Printer Software & https://www.makerbot.com/makerware/ \\ \hline
    SketchUp & 3D Modelling tool & http://www.sketchup.com/ \\ \hline
    SketchUp STL & STL Exporter plugin & http://extensions.sketchup.com/en/content/sketchup-stl \\ \hline
    \end{tabular}
\end{center}

\end{document}