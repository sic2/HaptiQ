\section{Abstract}

Today's technology is not as accessible for blind and visually impaired people.
Overcoming this digital divide gap for people with visual disabilities is one of today's most overlooked challenges.
Current technology provides either expensive or non-flexible devices. This dissertation investigates both commercial devices commonly used by visually impaired people, and the state of the art of haptic feedback devices.

This project presents the first vector-based display with haptic-audio feedback for people with visual disabilities: the HaptiQ. 
Two different devices of varying resolution were 3D-printed as prototypes.
In addition, a general easy-to-use API is provided together with two basic example applications. The first application shows how the HaptiQ can be used to explore simple graphs. The second program instead displays mathematical functions and is used to show how the HaptiQ facilitates edge recognition.
The sets of behaviour patterns used so far are heavily based on the use of Braille-like systems. 

With this project a new set of dynamic behaviour patterns is visually encoded in a way which favours the recognition of edges, corners and directions. 