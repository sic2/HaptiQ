\chapter{Conclusions}

Technology, in the last decade, has become increasingly ubiquitous. However, little effort has been spent on enabling the digital world to be used by people with disabilities. Blind people, in particular, have to face considerable difficulties when accessing textual and visual information. While screen readers are able to convey textual information with a good degree of success, the same is not true for data with a more graphical nature. Researchers are exploring haptic feedback technologies, allowing the blind to perceive shapes and textures, but low flexibility and high cost are still a main concern.  

This project presents the HaptiQ, an inexpensive haptic-audio feedback device that the visually impaired can use on interactive surfaces to explore both textual and graphical information. The main focus of the HaptiQ is to facilitate the blind to sense edges, corners and directions.

All the main objectives of this project have been achieved. Two prototypes have been created, an API provides both low and high level control over the device, and two example applications show some practical uses. This work presents novel approaches in the area of haptic feedback technology. The HaptiQ, to the best of my knowledge, is the first vector-based haptic feedback display for the visually impaired. No user study has been conducted to formally evaluate the performance of the device. However, informal observations have given positive feedback. The device also provides audio feedback, adding an additional dimension that users can perceive.

The HaptiQ uses a new set of dynamic tactons to convey information to the visually impaired. Tactons are ``structured, abstract, tactile messages which can be used to communicated information non-visually''\cite{brown2005first}. The tactile encoding used by the HaptiQ is strongly driven by the design of the device itself. The main emphasis, on designing both the device and the tactons as well, was to facilitate the recognition of edges, corners and directions by the visually impaired. 

This work has some limitations that will be addressed in the next versions of the HaptiQ. The first issue to address will be to re-engineer the device in order to decrease its size. A smaller device could allow many users to collaborate using the same interactive surface. Friction and texture are two dimensions that have not been explored due to time constraints. Nonetheless, these properties could open new unexplored perspectives.   

Moreover, the project opens to new areas of further research. A study targeting the usability and performance of the device itself would help the next development phase to produce a device closer to a final product. In addition, Régis Ongaro-Carcy and Dr Miguel A. Nacenta are currently working on building a strategic game for the visually impaired that makes use of the HaptiQ. A user study will evaluate how well the device performs when used collaboratively in a gaming environment. I have passively assisted the project by adapting both the hardware and the API to their needs. But, I am planning to actively contribute in the design and implementation of the game as well. 

Finally, the biggest contribution of this project has been trying to bridge the digital divide gap for the blind. This work will be released under the GNU General Public License allowing the visually impaired to make their own haptic device and applications at home, with a low budget. The HaptiQ is not a final product yet, but I am planning on improving both the hardware and the software components as a personal side project.
