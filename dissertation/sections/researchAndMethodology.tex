\chapter{Research and Methodology}

The nature of this project is more research than engineering oriented. It follows that common software development practices, such as SCRUM, were not used. Instead, the development process was articulated in three main phases: background research, hardware and software investigation, and hardware and software development. Each of these phases will be discussed below. 

\section{Project phases}
The background research phase was necessary to understand what the current state of the technology in haptic devices is. This phase was run mainly in the first two months of the project. 

The hardware and software investigation phase consisted in getting familiar with the 3D Printer, the Phidgets API and exploring what possible hardware could be used. For instance, hardware experimentation on Quantum Tunnelling Composite (QTC) was pursued to get pressure input from the device. The final evaluation was negative due to the reduced size of QTC tiles, which do not allow an easy installation on the HaptiQ. I then decided to use basic a Force-Sensing resistor instead. 

The hardware and software development phase followed the investigation phase logically, but not necessarily temporarily. Both the hardware and the software presented in this report are the product of an iterative exploratory design process. Weekly meetings with the supervisor were arranged. Each meeting consisted in reporting the work done in the previous week, planning the work for the week after and discussing design problems and features. These brainstorming meetings were fundamental in order to achieve the final HaptiQ.

Monthly meetings with Saad Attieh were arranged to discuss various design choices and get direct feedback from a blind person, which I found invaluable. 
One of the aims of this project is to develop an haptic device easy to build with an easy-to-use API. On the software side, the API has been designed having in mind that the code will be used for future work. Therefore, I wrote extensive XML documentation for the API calls and provided abstract instances, where necessary, to allow future users to extend the API. The attached manual contains a basic usage example of the API and the list of the required hardware components to 3D print and assemble the HaptiQ. 

\section{Management software}
Version control systems have been used to facilitate the software development of the API and maintain the 3D models of the HaptiQ always up-to-date. The Mercurial version control system, managed by the School of Computer Science, has been used from the beginning of the project until almost the very end. On the \nth{18} March 2014, with Régis Ongaro-Carcy joining the project, I decided to switch to GitHub \footnote{GitHub. \url{http://www.github.com/}. [Online; last checked: 03/04/2014].}. The main reasons behind this choice are that Régis does not have access to Mercurial repositories within the School of Computer Science and GitHub provides an issue tracking system that facilitates project work with more than one person.  

At the start of the project I used the issue tracker Bugify \footnote{Frondiz. Bugify. \url{http://www.bugify.com/}. [Online; last checked: 27/03/2014].}. However, later in the project development process I stopped using this tool because the project evolved too quickly and Bugify slowed down my process rather than speeding it up. 

\section{Testing}
One of the most challenging parts of this project was testing. Being the implemented API highly dependent on both the HaptiQ device (Phidget boards, servos, and pressure sensors) and the input device (table, camera, et cetera) the amount of Unit testing was minimal. This required me to manually ran the API multiple times and under various conditions in order to ensure that the software worked properly. A similar approach was undertaken to test the hardware. I regularly checked every part of the hardware to ensure that nothing was broken and that all pieces were installed correctly.
A simple logger was implemented to facilitate debugging.