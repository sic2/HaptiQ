\section{Abstract}

The use of technology for the blind and the visually impaired is not obvious. Overcoming this digital divide gap for people with visual disabilities is one of today's challenges. Current technology provides either expensive or not-flexible devices. Both commercial devices, used by visually impaired people, and the state of the art of haptic feedback devices are investigated.

The aim of this project is to build the first haptic-audio feedback vector-based display for people with visual impairments, which can be used on interactive surfaces: the HaptiQ. Two devices, with different resolutions, have been 3D-printed. In addition, a general easy-to-use API is provided together with two basic example applications. The first application shows how the HaptiQ can be used to explore simple graphs. The second program, instead, displays mathematical functions and is used to show the HaptiQ facilitates edges recognition.
The sets of behaviours used so far are heavily based on the use of Braille-like systems. With this project a new set of dynamic behaviours is presented for the first time, which favours the recognition of edges, corners and directions. 