\chapter{Design}

 The HTP provides the following properties through a single point haptic cue \todo[noline]{this paragraph is partially copied from paper} :
\begin{itemize}
	\item height and texture: "the relief and tactile feel that corresponds to elements displayed on the table".
    \item malleability: "how different materials respond to touch and pressure".
    \item friction: "resistance to movement in directions parallel to the table plane".
    \item location: "different positions on the table should provide appropriate haptic feedback".
    \item multiplicity: "the ability to provide multi-user or multi-point haptic feedback simultaneously". \cite{marquardt2009haptic}
\end{itemize}


\begin{enumerate}
	\item Hardware design
    	\begin{enumerate}
        	\item Improving the actuators
            	show pictures of old HTP and new MHTP
            \item pressure sensors (they are optional)
            \item actuators position facilitate directions. this is useful for blind people. an alternative device could provide actuators displaced as a grid.
            \item 4-MHTP
            \item 8-MHTP
            \item touch screen display SDK should be able to provide raw input
            \item from Brock's : "most blind seem to explore tactile maps with both hands and all 10 fingers"
            \item the computer needs a sound card for enabling text-to-speech output
        \end{enumerate}
	\item MHTP API architecture
    	\begin{enumerate}
        	\item state what architecture is targeted
        	\item Factory pattern with reflection in combination with delegation pattern was used to abstract the input from the touch screen device and notify the MHTP manager system
            \item Observer pattern is used to notify haptic shapes about any new input
            \item Bytetags (or glyphs?) were used to get input from touch screen device
            \item custom configuration
            \item API allows recognition of glyphs using different devices (touch displayes via surface sdk, webcam, etc)
            \item Limitations: there is a lower bound threshold on how often input can be acquired. this depends on the input device, the input technique used and the "refresh rate" of the MHTP
            \item texture recognition not impemented because servos cannot act over a certain frequency. Papert \cite{brown2005first} has a part where it says what's the skin frequency range. Solutions could include electro-vibration or haptic vibration motors (http://www.precisionmicrodrives.com/vibrating-vibrator-vibration-motors)
        \end{enumerate}
    \item MHTP API allows both high and low level manipulation of the device (in a similar fashion to HTP toolkit)
    \item Behaviour design
    \item Haptic objects
    \item Text exploration using SpeechSynthetiser - this can be personalised by adding custom haptic shapes which can react differently
\end{enumerate}


\textbf{NOTES:}

\begin{itemize}
    \item Ungar's paper --> conjoint retention hypothesis: combining spatial and linguistic information
    \item Ungar's paper --> "overall performance increases because the dual perceptual/linguistic representation provides a richer cueing and retrieval base for the learner to draw from during recall"
    \item Ungar's paper does not show any strong evidence that linguistic representation might add more to the perceptual one. However, that's a study from 2001
    \item MHTP can be classified as TUI but dynamic (see Ullmer 2005)
\end{itemize}

\missingfigure{add device picture}