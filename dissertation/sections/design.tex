\chapter{Design}

The HaptiQ aims to solve some of the problems identified in the related work. The hardware was developed using an iterative exploratory process. Therefore the design phases largely overlap with the implementation ones. The API provided has also undergone through many iterations. In this chapter I will first discuss the earliest prototypes. Then the 4-HaptiQ and the 8-HaptiQ hardware designs are illustrated. Finally, a detailed overview of the API is provided.

\section{The Hardware}
\subsection{Early Prototypes}
The hardware design of the HaptiQ has been initially based on the HTP design. Similarly to the HTP, the HaptiQ also uses mini-servos to control its actuators. However, while the HTP controls only one actuator, the HaptiQ controls four or eight of them. It immediately follows that as the number of actuators increases, so does the number of servos and the size of the device. In order to minimise the size, I designed three main prototypes by sketches (see Figure ~\ref{fig:HaptiQ-early-prototypes}). 

The first prototype extends the rotational axis in the xy plane (see Figure ~\ref{fig:first prototype}). The extension is designed as a gear that allows to choose combinations of three aligned actuators. An additional actuator would then be used to raise the selected actuators. This design reduces the number of servos, but all raised actuators will have the same height. Note that the focus of this design sketch was to minimise the number of servos, that is why actuators are shown as points, similar to Braille displays.

\begin{figure}
        \centering
        \begin{subfigure}[H]{0.5\textwidth}
                \includegraphics[width=\textwidth,height=\textwidth]{prototype1.png}
                \caption{Grid based}
                \label{fig:first prototype}
        \end{subfigure}%
        ~ %add desired spacing between images, e. g. ~, \quad, \qquad etc.
          %(or a blank line to force the subfigure onto a new line)
        \begin{subfigure}[H]{0.5\textwidth}
                \includegraphics[width=\textwidth,height=\textwidth]{prototype2.png}
                \caption{Gear based}
                \label{fig:second prototype}
        \end{subfigure}
        ~ %add desired spacing between images, e. g. ~, \quad, \qquad etc.
          %(or a blank line to force the subfigure onto a new line)
        \begin{subfigure}[H]{0.5\textwidth}
                \includegraphics[width=\textwidth,height=\textwidth]{prototype3.png}
                \caption{One servo per actuator}
                \label{fig:third prototype}
        \end{subfigure}
        \caption{HaptiQ early prototypes}\label{fig:HaptiQ-early-prototypes}
\end{figure}

In the second prototype I tried to minimize the total number of servos to two. One servo would be used to rotate a combinatorial gear about the z-axis(see Figure ~\ref{fig:second prototype}). A second servo can then be used to raise the first servo with the gear it controls and all the rods. This approach can be considered very versatile and could allow an high number of actuators by keeping the number of servos always to two. Nonetheless, the actuators would all be at the same heights when raised as in the first prototype. Since height can be used to convey additional information, this prototype was discarded and a third prototype was created.

The third prototype does not focus on minimising the number of servos any more, but rather on the possible functions that could be added to the HaptiQ. In this design each servo controls an actuator only and these are disposed circularly around the centre of the device (see Figure ~\ref{fig:third prototype}). The device would have a \textit{joystick} shape. Three pressure sensors could be added either at the bottom of the device or in between the internal and the external annuli. Using simple triangulation on the pressure sensors values, it could be possible to use the device like a gaming joystick and eventually augment its area of interaction. However, in one of the meetings with Saad, we realised that blind people would find this feature disrupting because understanding how far in the xy plane the device is augmented can become a challenging task. 

\subsection{4-HaptiQ}
The 4-HaptiQ is the first functional HaptiQ. This is an experimental project, with future work discussed in chapter 9, so I will occasionally refer to the device with the term prototype. This version has four actuators as shown in Figure ~\ref{fig:HaptiQ reference static actuators}. Unlike the third of the early prototypes, the main focus is on the mechanics and functions aspects.
The most successful aspect of the HTP, most probably, is the use of a small servo to mechanically move the rod along the z-axis. Nonetheless, the mechanic design used in the HTP has some flows that the HaptiQ tries to solve. The HTP, in fact, transforms the rotational motion of the servo to a linear motion only by ensuring that the servo works under a small range and that the rod slices outside the device through a small opening.
In the HaptiQ the actuators slide through a guide that allows motion only in the z-direction. In addition, each servo is linked to its actuator by using one or more screws inserted in between small openings in the actuator (see Figures ~\ref{fig:HaptiQ MinPos} and ~\ref{fig:HaptiQ MaxPos}). 

\begin{figure}
        \centering
        \begin{subfigure}[H]{0.3\textwidth}
                \includegraphics[width=\textwidth,height=\textwidth]{htp2.JPG}
                \caption{HTP}
                \label{fig:Mechanics HTP}
        \end{subfigure}%
        ~ %add desired spacing between images, e. g. ~, \quad, \qquad etc.
          %(or a blank line to force the subfigure onto a new line)
        \begin{subfigure}[H]{0.3\textwidth}
                \includegraphics[width=\textwidth,height=\textwidth]{pos1.JPG}
                \caption{HaptiQ MinPos}
                \label{fig:HaptiQ MinPos}
        \end{subfigure}
        ~ %add desired spacing between images, e. g. ~, \quad, \qquad etc.
          %(or a blank line to force the subfigure onto a new line)
        \begin{subfigure}[H]{0.3\textwidth}
                \includegraphics[width=\textwidth,height=\textwidth]{pos2.JPG}
                \caption{HaptiQ MaxPos}
                \label{fig:HaptiQ MaxPos}
        \end{subfigure}
        \caption{HTP and HaptiQ actuator mechanics}\label{fig:HTP and HaptiQ actuator mechanics}
\end{figure}

This version of the HaptiQ does not provide any case. The uneven surface leads to the lack of a reference point or plane for the actuators. This is a major problem in cases where the actuators change position while the user is not currently laying his hand on the device. In fact, it follows that when the blind user places the hand on top of the device again, he or her is not able to tell what the current state of the actuators is compared to the previous one. Two solutions are provided for the 4-HaptiQ: a mid-point reference and a plane reference static actuators (see Figure ~\ref{fig:HaptiQ reference static actuators}). It is not clear which type of reference actuator works best though. My hypothesis is that the plane reference actuator should be more comfortable when using the device and provides the same functionality than the point reference actuator. 

\begin{figure}
        \centering
        \begin{subfigure}[H]{0.5\textwidth}
                \includegraphics[width=\textwidth]{referencetype20.JPG}
                \caption{Mid-Point reference actuator}
                \label{fig:Mid-Point reference actuator}
        \end{subfigure}%
        ~ %add desired spacing between images, e. g. ~, \quad, \qquad etc.
          %(or a blank line to force the subfigure onto a new line)
        \begin{subfigure}[H]{0.5\textwidth}
                \includegraphics[width=\textwidth]{referencetype11.JPG}
                \caption{Plane reference actuator}
                \label{fig:Plane reference actuator}
        \end{subfigure}
        ~ %add desired spacing between images, e. g. ~, \quad, \qquad etc.
        \caption{HaptiQ reference static actuators}\label{fig:HaptiQ reference static actuators}
\end{figure}

The HaptiQ is the first vector-based display for blind users. Whether this design provides better feedback to users or not is unknown, because no study with participants could be conducted. Therefore, the actuators are designed so to support different tops in order to be able to evaluate the HaptiQ against a point-based haptic feedback TUI. In the version here presented, only vector-like tops are provided, but other type of tops can be easily printed and fasten to the actuators.   

Finally, the HaptiQ does not have any break to simulate friction. While it could be possible to add this feature, I decided that the primary goal of this project was just to focus on the vectorisation of the actuators. 

\subsection{8-HaptiQ}
The result of the final design iteration is the 8-HaptiQ. Using the same design of the 4-HaptiQ for eight actuators would lead to a considerable increase in size of the device. Therefore, the 8-HaptiQ has been completely re-engineered (see Figure ~\ref{fig:8-HaptiQ}). 

\begin{figure}[H]
  \centering
  \includegraphics[width=0.7\textwidth]{HaptiQ1.JPG}
  \caption{8-HaptiQ}
  \label{fig:8-HaptiQ}
\end{figure}

The 4-HaptiQ not only is not optimised in terms of space, but its wiring is also not well engineered. As it is possible to observe in Figure ~\ref{fig:HaptiQ reference static actuators} the wiring does not allow the device to be used comfortably, especially when one wants to rotate it. 
The servos in the 8-HaptiQ are organised in couples. Unlike its predecessor, the servos are positioned horizontally (see Figure ~\ref{fig:8-HaptiQservos}) and the screws are moved more toward the center of rotatory mechanism of the servos. These design choices allow the device to be more compact and still preserve the same functionalities of the 4-HaptiQ. 
The device is lift up by an additional structure laying below it, allowing the wiring to be less disruptive. 

\begin{figure}[H]
  \centering
  \includegraphics[width=0.7\textwidth]{8haptiQservos.JPG}
  \caption{8-HaptiQ servos couple}
  \label{fig:8-HaptiQservos}
\end{figure}

With the gathering of the wires for the sensors to the center of the device, the position of the sensors is also inverted compared to the 4-HaptiQ. While in the 4-HaptiQ the sensors meet in the middle, with this new version of the HaptiQ they are spread to the outside. It is not clear whether this design choice is an advantage or not in terms of usability. Alternatives to this solution will be provided in the near future, as discussed in the Future Work chapter.

The 8-HaptiQ takes a different approach for the design of the actuators guides. In the first version, a base was provided for the actuator, which limits this to move further down when this is at its minimum position (see Figure ~\ref{fig:HaptiQ MinPos}). But to provide an optimal wiring, the 8-HaptiQ combines the guides for all the actuators with the reference plane (see Figure ~\ref{fig:8haptiqref}). Additional connecting screws allow the actuators to be always controlled by the servos and not to fall below its minimum position. 

\begin{figure}[H]
  \centering
  \includegraphics[width=0.6\textwidth]{8HaptiQSlots2.png}
  \caption{8-HaptiQ reference plane and actuators slots}
  \label{fig:8haptiqref}
\end{figure}

\subsection{Fiducial Markers}

The HaptiQ, like the HTP, is designed to work on interactive tabletops. Specifically on Microsoft PixelSense tables (supporting Microsoft Surface SDK 2.0) \cite{pixelsense}. The Microsoft Surface Bytetag is the standard fiducial marker for the PixelSense tables (see Figure ~\ref{fig:Bytetag}). But Bytetags are hard to print, if high accuracy is wanted, and expensive, if one wants to buy them from the official store.

The HaptiQ, in addition to the  Microsoft Bytetags, supports Glyph, as defined by the Glyph Recognition And Tracking Framework (GRATF) (see Figure ~\ref{fig:glyph}).

Both types of fiducial markers are to be displaced on the bottom of the device, if a PixelSense table (or equivalent) is used. Otherwise, it could be possible to place a Glyph on top of the device, provided that an appropriate structure is built, and recognise it using a web-cam. 

\begin{figure}[H]
        \centering
        \begin{subfigure}[H]{0.3\textwidth}
                \includegraphics[width=\textwidth]{bytetag.jpg}
                \caption{Bytetag}
                \label{fig:Bytetag}
        \end{subfigure}%
        ~ %add desired spacing between images, e. g. ~, \quad, \qquad etc.
          %(or a blank line to force the subfigure onto a new line)
        \begin{subfigure}[H]{0.3\textwidth}
                \includegraphics[width=\textwidth]{glyph3.png}
                \caption{Glyph}
                \label{fig:glyph}
        \end{subfigure}
        ~ %add desired spacing between images, e. g. ~, \quad, \qquad etc.
        \caption{Fiducial Markers}
        \label{fig:fiducialMarkers}
\end{figure}

\section{The API}

This section discusses the design choices taken for the API of the HaptiQ. The general idea is similar to the HapticTouch toolkit \cite{ledo2012haptictouch}. However, this has been completely redesigned to support:

\begin{itemize}
	\item multiple actuators
    \item multiple pressure sensors
    \item a completely new approach to HapticObjects and HapticBehaviours
    \item actions called when input is received by the user
    \item additional API to get input position of the device from different hardware
\end{itemize}

Each of these will be discussed in the following sections. 

\subsection{Overall Structure}

The API consists of two main components: the HaptiQ API used to control the device and the Input API (see Figure ~\ref{fig:api}). 
The HaptiQ API component is designed to provide high and low access to the HaptiQ. This allows the API to be versatile and to be used by beginner, intermediate and advanced developers. 
The HaptiQ API retrieves the position of the device through the Input API. 

The following two sections describe these components in more details.

\begin{figure}[H]
  \centering
  \includegraphics[width=1.0\textwidth]{General_overview.png}
  \caption{API General Overview}
  \label{fig:api}
\end{figure}

\subsection{Input API}

Initially, the HaptiQ, like the HTP, was designed to work on a Microsoft PixelSense table only \cite{pixelsense}.
Microsoft Surface Bytetags are located underneath the HTP to get the location of the device in the table. However, official Bytetags are expensive and printing them with a normal printer does not lead to satisfying results. In addition, the HaptiQ will be used in the near future on digital tables that do not support the Surface SDK. Therefore, in order to abstract the HaptiQ API from any hardware input dependence, a separate API component has been created (see Figure ~\ref{fig:input_api}). 

The architectural structure of the Input API is based on the classic Factory pattern. This adds a level of abstraction over the hardware that should be used to get the position of the HaptiQ. Currently the API supports Bytetags and Glyphs recognition. But client applications can also add their own customised class to support different position detection methods. 

\begin{figure}[H]
  \centering
  \includegraphics[width=0.5\textwidth]{input.png}
  \caption{Input API}
  \label{fig:input_api}
\end{figure}

\subsection{HaptiQ API}

The HaptiQ API plays the main role within the API by connecting the Input API, the HaptiQ device and the client applications all together (see Figure ~\ref{fig:haptiQ_api}). 

The HaptiQsManager is designed using the Singleton pattern, ensuring that only one instance exists, and controls the different parts of the internal structure of the HaptiQ API. On creation, the manager launches the configuration manager. At this stage, the API looks in the current directory for pre-saved configurations of HaptiQs. If there are not valid configurations matching the current attached devices to the machine, a configuration windows form is started, allowing users to set up devices. Otherwise, HaptiQ objects are created for each configured device. 

Each HaptiQ is a thread process on its own, which runs the current behaviours cyclically. The HaptiQ class is responsible both for controlling the actuators and getting the input via the pressure sensors. 

The manager, however, is responsible to get the devices position through the Input API. Using the observer pattern, the HapticObjects registered by the client applications are notified whenever a new input is received. If the input is relevant to a particular HapticObject, then a behaviour is returned to the HaptiQ device that caused it. 

The API allows behaviours to be added, removed, and substituted. This means that the current state of a device can be the combination of multiple behaviours. This adds an extra dimension to the haptic cues that can be transferred to the user. 

On the other hand, each HaptiQ can keep track of pressure events, without the intervention of the HaptiQsManager. Pressure data events are automatically fired to registered HapticObjects. Each HapticObject reacts differently to pressure input. 

\begin{figure}[H]
  \centering
  \includegraphics[width=1.0\textwidth]{haptiQAPI.png}
  \caption{HaptiQ API}
  \label{fig:haptiQ_api}
\end{figure}

\subsubsection{Haptic Objects}
Basic haptic objects are provided by the API and are designed to be used by WPF applications without having to concern about behaviours and additional low level mechanisms. All haptic objects can carry textual information. In fact, applications should prioritise the information one wants to convey, rather than the aesthetics of the objects. Therefore, only simple shapes are provided. Obviously, one can extend the HapticShape abstract class to provide more interesting shapes, but this is outside the scope of this project.

Haptic objects are the main visual components of the client applications. Therefore, clients can also register custom actions to be run when pressure events are received by a specific object.

Figure ~\ref{fig:hapticObjects} shows examples of the haptic objects supported by the API. Rectangles and circles can be used as identity objects, in applications like graphs. Identity objects can also be linked. Links can have directions (from red to green) or not. The API allows also to specify lines and polylines. One of the example applications provided with the API, for example, uses polylines to display mathematical functions.

The used Fiducial Markers do not provide very precise input location of the device. In order to compensate, the HapticObjects provided are extended (red borders in Figure ~\ref{fig:hapticObjects}). This makes it harder to the user to leave the object, compensating for the inaccuracy of detecting the device location.

\begin{figure}[H]
        \centering
        \begin{subfigure}[H]{0.4\textwidth}
                \includegraphics[width=\textwidth]{objects.png}
                \caption{Identity objects and links}
                \label{fig:objects1}
        \end{subfigure}%
        ~ %add desired spacing between images, e. g. ~, \quad, \qquad etc.
          %(or a blank line to force the subfigure onto a new line)
        \begin{subfigure}[H]{0.4\textwidth}
                \includegraphics[width=\textwidth]{objects2.png}
                \caption{Line objects}
                \label{fig:objects2}
        \end{subfigure}
        ~ %add desired spacing between images, e. g. ~, \quad, \qquad etc.
        \caption{Haptic Objects}\label{fig:hapticObjects}
\end{figure}

\subsubsection{Audio Feedback}

The audio component is a standalone one, allowing any part of the API, as well as client applications, to output beeps or textual information as audio.

While this is a small component, it still plays an important role in the API. Ungar and Blades \cite{ungar2000can} discuss whether the conjoint retention hypothesis, combining spatial and linguistic information, can provide richer haptic cueing. Their study did not show any significant improvement when adding audio feedback. However, they results were not definitive. Therefore, I decided to follow the trend of many commercial applications, like VoiceOver, which use audio to provide textual and position feedback too.

In the API provided, the Beep feedback is used only for HapticLines and HapticPolylines. The reason is that following lines is harder than identifying the existence of shapes. This approach is similar to the one taken in audio feedback applications, such as VoiceOver.

The speech synthesiser, instead, is used to read out textual information stored in the HapticObjects.

\section{Tactons}
\label{sec:designTactons} 

Pietrzak et al. \cite{pietrzak2009creating} propose a set of tactons that can be used on pin array devices and that could be used to indicate edges and directions. These tactons are designed for a 4x4 pins grid. The HaptiQ provides a maximum of eight actuators, uses lines rather than points to convey haptic cues and can provide height information as well (pin arrays typically provide only 0/1 information). Here, I present a new set of tactons. In this section the terms tacton and behaviour will be used interchangeably.

A tacton is used by an HapticObject to notify a device how to react when this is located on top of it. In the provided API there is a strong relationship between each HapticObject and a behaviour. Nonetheless, behaviours can be used by client defined HapticObjects or directly from a client application.

In the following sections, the scales in Figure ~\ref{fig:scales} will be used to represent the heights of the actuators.

\begin{figure}[H]
        \centering
        \begin{subfigure}[H]{0.7\textwidth}
                \includegraphics[width=\textwidth]{scaleA.png}
                \caption{4-HaptiQ}
                \label{fig:scale-4}
        \end{subfigure}%
        ~ %add desired spacing between images, e. g. ~, \quad, \qquad etc.
          %(or a blank line to force the subfigure onto a new line)
          
        \begin{subfigure}[H]{0.7\textwidth}
                \includegraphics[width=\textwidth]{scaleB.png}
                \caption{8-HaptiQ}
                \label{fig:scale-8}
        \end{subfigure}
        ~ %add desired spacing between images, e. g. ~, \quad, \qquad etc.
        \caption{Blue-Red scale used to indicate the height of the actuators in tactons diagrams. Numerical values represent the positions of the actuators in percentage}
        \label{fig:scales}
\end{figure}

\subsection{Flat and Max}

Flat and Max are the simplest tactons of the set (see Figure ~\ref{fig:Basictactons}). The Flat tacton is typically used to indicate that underneath the device there are no objects or information. The Max behaviour, instead, can be used to represent the existence of an object. For instance, when the device is inside an HapticRectangle the Max behaviour is played. 

\begin{figure}[H]
  \centering
  \includegraphics[width=0.5\textwidth]{basic.png}
  \caption{Basic tactons: Flat and Max}
  \label{fig:Basictactons}
\end{figure}

\subsection{Notification}
\label{sec:notification}

The Notification tacton, similarly to the Max tacton, can also be used to notify an object. In the case being, this tacton is used to notify an HapticCircle. This tacton consists into cyclically changing the positions of the actuators (see Figure ~\ref{fig:notification}). Each rod always increases or decreases its height position. But in the case the maximum or minimum height threshold is reached, the actuator direction (increase or decrease in height) is inverted.

\begin{figure}[H]
        \centering
        \begin{subfigure}[H]{0.7\textwidth}
                \includegraphics[width=\textwidth]{notification-4.png}
                \caption{4-HaptiQ}
                \label{fig:notification-4}
        \end{subfigure}%
        ~ %add desired spacing between images, e. g. ~, \quad, \qquad etc.
          %(or a blank line to force the subfigure onto a new line)
          
        \begin{subfigure}[H]{0.7\textwidth}
                \includegraphics[width=\textwidth]{notification-8.png}
                \caption{First 12 cases for 8-HaptiQ}
                \label{fig:notification-8}
        \end{subfigure}
        ~ %add desired spacing between images, e. g. ~, \quad, \qquad etc.
        \caption{Notification tacton}\label{fig:notification}
\end{figure}

\subsection{Edges and Corners}

The Edge-Corner tacton is probably the most significant one of all the tactons by allowing users to perceive edges and corners. 

Figure ~\ref{fig:edges} show how the combinations of actuators, in both the prototypes, are used to represent edges. The resolution of the device is increased by alternatively actuating adjacent actuators. It is possible to achieve a precision of 45° and 22.5° for the 4-HaptiQ and the 8-HaptiQ respectively. 

\begin{figure}[H]
        \centering
        \begin{subfigure}[H]{0.3\textwidth}
                \includegraphics[width=\textwidth]{direction-4.png}
                \caption{4-HaptiQ}
                \label{fig:direction-4}
        \end{subfigure}%
        ~ %add desired spacing between images, e. g. ~, \quad, \qquad etc.
          %(or a blank line to force the subfigure onto a new line)
        \begin{subfigure}[H]{0.5\textwidth}
                \includegraphics[width=\textwidth]{direction-8.png}
                \caption{8-HaptiQ}
                \label{fig:direction-8}
        \end{subfigure}
        ~ %add desired spacing between images, e. g. ~, \quad, \qquad etc.
        \caption{Direction tacton - Edges}\label{fig:edges}
\end{figure}

It is also possible to represent corners, which are merely the combinations of two edges. The 4-HaptiQ can represent corners of 90° with a precision of 45° (see Figure ~\ref{fig:corner-4}). Indeed, the 8-HaptiQ can represent corners of 45°, 90° and 135°, with a precision of 22.5° (see Figures ~\ref{fig:corner-8}, ~\ref{fig:corner-8c}). 

\begin{figure}[H]
        \centering
        \begin{subfigure}[H]{0.5\textwidth}
                \includegraphics[width=\textwidth]{corners-4.png}
                \caption{4-HaptiQ}
                \label{fig:corner-4}
        \end{subfigure}%
        ~ %add desired spacing between images, e. g. ~, \quad, \qquad etc.
          %(or a blank line to force the subfigure onto a new line)
          
        \begin{subfigure}[H]{0.4\textwidth}
                \includegraphics[width=\textwidth]{corners-8.png}
                \caption{8-HaptiQ}
                \label{fig:corner-8}
        \end{subfigure}
        ~ 
         \begin{subfigure}[H]{0.4\textwidth}
                \includegraphics[width=\textwidth]{corners-8c.png}
                \caption{8-HaptiQ (combinations)}
                \label{fig:corner-8c}
        \end{subfigure}
        ~ %add desired spacing between images, e. g. ~, \quad, \qquad etc.
        \caption{Direction tacton - Corners}\label{fig:corners}
\end{figure}

The Edge-Corner tacton is used by the HapticRectangle, to represent its borders, and the HapticLine and HapticPolyline objects. 

\subsection{Pulsation and Linear}
The Pulsation and Linear tactons are used with the HapticLink to represent the sign of the link between two Haptic objects. Figure ~\ref{fig:pulsation} shows two cases of the pulsation tacton for the 4-HaptiQ. By symmetry it is possible to represent 8 (or 16 for the 8-HaptiQ) directions. This tacton also allows frequency to be specified, so that pulsation increases as one gets closer to the target. The static actuators in this tacton are also fundamental to identify orientation of the direction.

The Linear tacton, on the other hand, is the union of the pulsation tactons, but without any pulsation behaviour. Thus, the height of the actuators is used rather than the pulsation frequency in order to convoy different information about a direction. For example, the tacton can be encoded so that the height of the actuators increases linearly as the distance between the device and the target approaches zero.

\begin{figure}[H]
  \centering
  \includegraphics[width=0.5\textwidth]{pulsation.png}
  \caption{Pulsation tacton}
  \label{fig:pulsation}
\end{figure}

\section{Applications}

The HaptiQ can provide a new set of haptic cues which allow the visually impaired to better perceive lines, edges and directions. The two example applications shown below illustrate the types of applications that can be created using the HaptiQ API. Note that these applications have been designed as proof-of-concepts and not as applications ready to be deployed to the end users.

\subsection{Graphs Visualiser}
\label{sec:graphVis}

The first application is a basic graph visualiser. The application is designed to be used by people with visual disabilities, with the assistance of a person with normal or corrected vision. The assistant can create graphs using the provided graphical user interface (see Figure ~\ref{fig:graphsVis}). Graphs contain identity objects and links as well as lines and polylines. \\
In order to create new object, the assistant has to select one of the options on the left hand side. A new window will be prompted asking for the properties of the haptic object. The procedure is the same for all the objects, except for links where the user has to select which objects to link by physically touching the objects with a finger or a bytetag. 

The design is not complete, since it does not allow objects to be deleted and graphs to be saved for later use. However, this is a good example of a possible application for the HaptiQ. One of the issues raised by Saad, in fact, is that blind people have difficulties in visualising graphs such as Finite Automata or Trees. 

\begin{figure}[H]
  \centering
  \includegraphics[width=0.5\textwidth]{graph1.png}
  \caption{Graphs Visualiser}
  \label{fig:graphsVis}
\end{figure}

\subsection{Functions Visualiser}
\label{sec:funcVis}

The second application that this project uses to prove the flexibility and usability of the HaptiQ and its API is a function visualiser. One of the most challenging tasks for the visually impaired is to be able to \textit{visualise} mathematical functions. The design of the function visualiser is also very simple. The user inputs a function and the application then evaluates and displays it (see Figure ~\ref{fig:funcVisualiser}). The application uses haptic lines and polylines to represent the functions. Cartesian axes are also added for completeness.

\begin{figure}[H]
        \centering
        \begin{subfigure}[H]{0.4\textwidth}
                \includegraphics[width=\textwidth,height=\textwidth]{functions1.png}
                \caption{Linear function}
                \label{fig:funct1}
        \end{subfigure}%
        ~ %add desired spacing between images, e. g. ~, \quad, \qquad etc.
          %(or a blank line to force the subfigure onto a new line)
        \begin{subfigure}[H]{0.4\textwidth}
                \includegraphics[width=\textwidth,height=\textwidth]{functions2.png}
                \caption{Sine function}
                \label{fig:funct2}
        \end{subfigure}
        \caption{Functions Visualiser}\label{fig:funcVisualiser}
\end{figure}
