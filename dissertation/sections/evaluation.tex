\chapter{Evaluation}

\begin{itemize}
	\item project can be considered successful
    \item achieved most of objectives
    \item provided an extensive API
    \item done more work than planned for certain aspects
    \item project very open ended
    \item feedback from saad
    \item feedback from Regis
    \item discuss some of the achievements
    \item discuss some of the issues and how they could be solved
    \item compare this work with the one in the background chapter
\end{itemize}


-----

\begin{itemize}
    \item alternative actuators positions (see Piatrzak paper: Creating usable pin array tactons for nonvisual information)
    \item can use hidden-markov-model or time-series-analysis to identify patterns in pressure input and classify gestures
    \item evalute the learning rate for blind people in terms of behaviours of the device
    \item possible experiment --> use of MHTP with audio VS audio only (like VoiceOver in iphone)
\end{itemize}

The HTP provides the following properties through a single point haptic cue:
\begin{itemize}
	\item height and texture: "the relief and tactile feel that corresponds to elements displayed on the table".
    \item malleability: "how different materials respond to touch and pressure".
    \item friction: "resistance to movement in directions parallel to the table plane".
    \item location: "different positions on the table should provide appropriate haptic feedback".
    \item multiplicity: "the ability to provide multi-user or multi-point haptic feedback simultaneously". \cite{marquardt2009haptic}
\end{itemize}

With the HaptiQ all these features are addressed, except texture and friction. 

texture recognition not impemented because servos cannot act over a certain frequency. Papert \cite{brown2005first} has a part where it says what's the skin frequency range. Solutions could include electro-vibration or haptic vibration motors (http://www.precisionmicrodrives.com/vibrating-vibrator-vibration-motors)
