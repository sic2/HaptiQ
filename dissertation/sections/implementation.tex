\chapter{Implementation}

\section{The Hardware}

Designed in Sketchup and printed using MakerBot\textregistered  Replicator\texttrademark 2x (experimental 3d printer)

\section{The API}

\subsection{Input API}

\begin{itemize}
	\item targeted architecture
    \item bytetags, glyphs
\end{itemize}

\subsection{HaptiQ API}

\subsection{Examples}

The API, as previously state, has been designed and implemented so that even beginners can easily use it. Code ~\ref{lst:basicAPIUsage} shows a very basic example of its usage. 


\lstset{style=sharpc}
\begin{lstlisting}[caption={Basic API usage},label={lst:basicAPIUsage}]
using MHTP_API;
using HapticClientAPI;

public partial class Application : SurfaceWindow
{
  public Application()
  {
      MHTPsManager.Create(windowTitle, "SurfaceInput");
      
      HapticShape rect0 = new HapticRectangle(50, 50, 150, 200);
      rect0.color(Brushes.Salmon);
      this.ContainerTest.Children.Add(rect0);
      
      HapticShape rect1 = new HapticRectangle(150, 350, 200, 200);
      rect1.color(Brushes.Orange);
      this.ContainerTest.Children.Add(rect1);
      
      HapticShape rect2 = new HapticRectangle(550, 150, 100, 100);
      rect2.color(Brushes.Green);
      this.ContainerTest.Children.Add(rect2);
      
      HapticShape link = new HapticLink(rect2, rect1, false);
      link.color(Brushes.White);
      this.ContainerTest.Children.Add(link);
  }
}
\end{lstlisting}

\section{Configuration}

\section{Tactons}

\section{Applications}



