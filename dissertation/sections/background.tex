\chapter{Background}

In the following chapter the state of the art in haptic feedback technologies is presented, in addition to the current commercial technology for the visually impaired. The related work taken into consideration is taxonomised based on how the various techniques and devices work and how users interact with them.

Research in haptic perception and visualization has grown significantly in the last 25 years \cite{roberts2007we}. Being able to cover most of the work in haptics is not feasible. Therefore, only the most significant related work to this project will be addressed.\todo[noline]{Read background section of HTP paper again}

\section{Commercial technology for the visually impaired}
The devices and techniques currently available on the market and used by most of the visually impaired differs from what researchers have been able to achieve in the last decade. The main reason is that the most advanced technology produced by researchers is either expensive or still a prototype, or both. 

Traditional Braille and tactile maps are by far the most intuitive and easy-to-use ways to access textual and graphical information. However, these approaches are static and do not allow access to electronical devices. The resolution of tactile maps is usually low, since font size tends to be large and zoom-in actions are not allowed. Dynamic Braille displays \cite{HumanWare, shimada2010development, blindMaps} allow traditional techniques to be used for technological devices. However, only a small portion of the display can be covered using these devices and being able to keep track of the current position within the display becomes challenging.

The auditory system for blind people is probably the most important of their senses. Using the tactile system to perceive the surrounding world, in fact, can be hard to learn, especially for people who become blind late in their life \todo[noline]{rephrase}. Screen readers, like JAWS, or auditory-aiding software like VoiceOver\cite{voiceOver} allow the visually impaired to explore textual information and navigate maps as well. Systems that rely just on audio feedback allow a user to follow lines and directions by emitting "positive" or "negative" sounds. It obviously follows that a user needs to constantly explore the area around a specific point in the map in order to perceive the direction of a line or understandig the shape of an object.    
    
\section{Multimodal Interactive Maps}
An attempt to allow blind people to access geographical maps through digital devices is via multimodal interactive maps (MIMs). MIMs are input-output devices that can make use of different technologies to allow user to access or insert information (i.e Input: speech recognition, touch input. Output: audio, tactile feedback). The work presented by Broke et al \cite{brock2010usage} is the most recent type of MIM, to my knowledge. People who are visually impaired usually tende to use all their fingers when exploring objects. Therefore, enhancing multitouch input allows users to perceive maps with greater ease. Nonetheless, the maps have to be printed for each new visualisations. 

\section{Tangible User Interfaces with Mechanical Actuators}
Haptic feddback devices that use mechanical actuations are becoming increasingly popular. In fact, these are usually more flexible and adaptable to displays of different sizes. They also provide better tactile cues than audio-based techniques and MIMs.

The first device to take in consideration is the Haptic Tabletop Puck (HTP)\cite{marquardt2009haptic}, which the HaptiQ can be considered the successor of. The HTP is a simple, inexpensive and small haptic device that suits particularly well for use on digital tabletop surfces. The tactile cue, however, is applied only through one single point. The HTP has not been designed specifically for visually impaired people. Therefore, it is not very suitable for more complex tasks such as recognising a line or a direction and follow it. 

This type of problem has been observed also with the SensAble PHANToM\cite{massie1994phantom, yu2001haptic}. Participants were asked to follow one or more lines. But the experiment showed negative results, since having only one point contact did not allowed the participants to well understand the shape of the line, especially at the end points or corners. Similarly to users of VoiceOver, blind people using the PHANToM or the HTP need to guess the direction of lines and eventually randomly explore the surrounding. 

Harrison and Hudson \cite{harrison2009providing} propose dynamically changeable physical buttons that use either pneumatic actuation or Shape Memory Materials to provide enhance displays with tactile feedback. This type of displays is proved to be cheap, easy to produce and effective. They are also convenient for static public displays (i.e. ATMs) or cars panel control, but not for more generic displays, such as tabletops, since specific masks need to be created for each visualisation. 

Recently the Tangible Media Group, MIT, has presented inFORM system, a 2.5D shape display that uses mechanical actuators to provide feedback to the user\cite{follmer2013inform}. Top-down projection and a kinect are used to provide visual feedback and acquire user input. inFORM provides excellent haptic feedback and equivalent responsiveness, nonetheless the table is bulky and expensive. The major limitation of these techniques is that only one tactile signal can be transmitted to a surface. 

\section{Haptic Feedback by alternative mediums}

Recently Bau et al. presented TeslaTouch\cite{bau2010teslatouch} and REVEL\cite{bau2012revel}, two haptic feedback systems based on electrovibration and reverse electrovibration respectively. The purpose of these systems is to create Augmented Reality Tactile displays. These could be both extrinsic, integrated in the environment, or intrinsic, augmenting the user. A study on visually impaired people, using the TeslaTouch, has also been conducted \cite{xu2011tactile}. Participants found hard to recognise dots, as used in Braille, especially due to the fact that the tactile feedback is perceived only when the finger moves. Straight lines could be recognised, allowing some shapes be perceived (circle, square and triangle), but not with accurate precision. Another limitation of vibrotractile based devices is that it cannot be used by multiple users concurrently. 

Other techniques used to improve haptic feedback on touchscreens rely on electromagnetism. FingerFlux \cite{weiss2011fingerflux} uses magnets attached on fingertips to provide "attraction, repulsion, vibrations and directional haptic feedback" in a table with electromagnets pixels. The main issue related to FingerFlux is that the resolution of the display cannot be increased over a certain threeshold, otherwise the electromagnets interfere with each other. In addition, FingerFlux allows to create horizontal forces, but it is not possible to rendere sharp surfaces.    

UltraHaptics provides multi-point haptic feedback above an interactive surface using acoustic radiation force to generate haptic cues on targets in mid-air \cite{carter2013ultrahaptics}. The system has been shown to be fairly accurate and easy to use, yet accuracy degrades quickly as the user's hand is above 200mm. 

Haptic feedback can also be transmitted by ejecting air at high pressures. AIREAL is a scalable, inexpensive and practical system that uses air to augment reality of interactive surfaces and environments \cite{sodhi2013aireal}. AIREAL works well within gaming environments and haptic cues of external environment, but the low resolution limits its use for objects and edges recognition, especially for the visually impaired.