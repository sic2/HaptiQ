\chapter{Background}

In the following chapter the state of the art in haptic feedback technologies is presented, in addition to the current commercial technology for the visually impaired. The related work taken into consideration is taxonomised based on how the various techniques and devices work and how users interact with them.

Research in haptic perception and visualization has grown significantly in the last 25 years \cite{roberts2007we}. Being able to cover most of the work in haptics is not feasible. Therefore, only the most significant related work to this project will be addressed.\todo[noline]{Read background section of HTP paper again}

\section{Commercial technology for the visually impaired}
The devices and techniques currently available on the market and used by most of the visually impaired differs from what researchers have been able to achieve in the last decade. The main reason is that the most advanced technology produced by researchers is either expensive or still a prototype, or both. 

Traditional Braille and tactile maps are by far the most intuitive and easy-to-use ways to access textual and graphical information. However, these approaches are static and do not allow access to electronical devices. The resolution of tactile maps is usually low, since font size tends to be large and zoom-in actions are not allowed. Dynamic Braille displays \todo[noline]{cite here or in intro?} allow traditional techniques to be used for technological devices. However, only a small portion of the display can be covered using these devices and being able to keep track of the current position within the display becomes challenging.

The auditory system for blind people is probably the most important of their senses. Using the tactile system to perceive the surrounding world, in fact, can be hard to learn, especially for people who become blind late in their life \todo[noline]{rephrase}. Screen readers, like JAWS, or auditory-aiding software like VoiceOver\cite{voiceOver} allow the visually impaired to explore textual information and navigate maps as well. Systems that rely just on audio feedback allow a user to follow lines and directions by emitting "positive" or "negative" sounds. It obviously follows that a user needs to constantly explore the area around a specific point in the map in order to perceive the direction of a line or understanding the shape of an object.    
    
\section{Multimodal Interactive Maps}
An attempt to allow blind people to access geographical maps through digital devices is via multimodal interactive maps (MIMs). MIMs are input-output devices that can make use of different technologies to allow user to access or insert information (i.e Input: speech recognition, touch input. Output: audio, tactile feedback). The work presented by Broke et al \cite{brock2010usage} is the most recent type of MIM, to my knowledge. People who are visually impaired usually tende to use all their fingers when exploring objects. Therefore, enhancing multitouch input allows users to perceive maps with greater ease. Nonetheless, the maps have to be printed for each new visualisations. 

\section{Feedback by Mechanical Actuators}

\subsection{Changeable Physical Buttons}

Major problem: display is static. Must be created in advance.

\subsection{inFORM}

The Tangible Media Group, MIT, has recently presented inFORM at UIST13. 

(see: \cite{follmer2013inform})

\subsection{Microsoft 3D Haptic}

http://research.microsoft.com/en-us/news/features/3-dhaptic-060413.aspx

\section{Haptics by Electrovibration}


\subsection{Revel}

Predecessor is TeslaTouch - revel works the opposite direction

Introducing the concept of ARTactile displays: 
\begin{enumerate}
	\item Extrinsic haptic: integrated in the environment
    \item Intrinsic haptic: augment the user
\end{enumerate}

\subsection{FingerFlux}

\section{Haptics through air}
\subsection{Ultrahaptics}
\subsection{AIREAL}

\section{TUIs}

\subsection{Madgets}
\subsection{PICO}
\subsection{slap widgets}

@see inFORM paper Related work section for more

\section{Conclusion/Summary}
The understanding of the current state of haptic feedback technology is crucial to understand how the MHTP tries to solve some of the limitations of other devices and how it could be improved (\textbf{see Evaluation chapter}). 